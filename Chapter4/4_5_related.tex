\section{Related Work} 
\label{sec:4_5_related}

%ONE-WAY CS
Prior works on one-way car-sharing services revealed some important characteristics of these services as its usage patterns and their impact on the urban centers~\cite{becker:17, ciari2014modeling, martin2011impact, boldrini:16}. For example, one-way car-sharing systems are mostly used in dense urban areas with good public transportation system~\cite{stillwater2009carsharing}. Young people with a higher education level are more attracted to use this service~\cite{burkhardt2006attracted}.  Moreover, several works also confirm positive impacts on the actual transport system, such as the reduction on traffic and emission of pollutants~\cite{cervero2004city,martin2011impact}, the increase of free parking spots and in the use of public transport~\cite{shaheen2010carsharing}.
These prior works also reveal that one-way car-sharing services are used for long journeys and shopping~\cite{ciari2014modeling}. In most cases, at least two passengers use the vehicle~\cite{becker:17}. Finally, these works also reveal interesting features about the fleet of electric cars. For instance, vehicles remain parked in central regions for lower periods than in suburban regions, directly impacting the autonomy of the vehicles~\cite{boldrini:16}.

%FREE-FLOATIN CS
Previous works also point out the differences between the free-floating and the one-way model services. Indeed, the free-floating vehicles are often used for shorter periods, presenting commuting trips and a considerable number of trips to airports~\cite{ciari2014modeling}, \cite{becker:17} \cite{coccacar}. Typically, free-floating vehicles carry a single user~\cite{becker:17} and this user presents fast driving habits~\cite{ciociolaumap}. Finally, the free-floating model also presents a periodical usage: during the mornings, central areas of the city are the main destination, while during the evening, suburban areas are reached more~\cite{ciociolaumap}. 
Despite the flexibility of the free-floating and one-way model, previous works have not observed a clear difference in users preferences between them~\cite{ciari2014modeling}. On the other hand, some works have identified that these services attract different users classes, exposing the fact that free-floating models and station-based models must be treated separately~\cite{becker:17}.

%OUR PRIOR WORK AND DIFFERENCES
To the best of our knowledge, only our prior works characterize the  two-way car-sharing service model~\cite{rooke2018caracterizacao,rooke:19}. 
More precisely, in \cite{rooke2018caracterizacao} we first characterize the usage patterns and the demands of \textit{Modo},\footnote{http://www.modo.coop/} a car-sharing service that operates in Vancouver (Canada) and nearby regions. We present a simple model that represents the demand for vehicles in this car-sharing system, presenting statistical analysis to parametrize this model. Then, in \cite{rooke:19}, we further explore this two-way car-sharing service model, by evaluating two distinct periods and also present a spatial analysis of the vehicle demands. Our results evidence long travel duration, and many cancellations which produce a low utilization factor of the system. Moreover, the two-way system usage presents a strong relationship with the public transport system, as well as with regions nearby points of interests, such as public universities and commercial centers~\cite{rooke:19}.  In~\cite{coccaopt,coccacar} we analyzed free-floating car-sharing data in different cities and propose models and optimization methods in order to efficiently use electric cars.  
%In the current work, we expand our studies, taking into account simultaneously the three car-sharing. %categories in the same city. %We study these services, characterizing their user behavior (i.e., the demand and usage patterns) observing the main similarities and differences between them. Moreover, in the current work, we jointly provide a spatial-temporal analysis and, we believe the data we provide and the models we develop can be useful to generate accurate synthetic workload which in turn, can be used as a substrate to the simulation and the design of new systems.
%In sum, we provide an in-depth characterization of car-sharing services, including service temporal, spatial-temporal and, user behavior characteristics. 
We are not aware of studies that jointly study the three types of services in the same city, leveraging their common characteristics and highlighting its particularities as we are doing in the present work.