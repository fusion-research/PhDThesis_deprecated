\section{Introduction}
\label{sec:4_1_intro}

Urban mobility is a key research area, attracting several academic studies and private investments. It is intrinsically connected to a wide number of urban activities, such as the demand for communication resources. Understanding the urban mobility, specifically the traffic-related mobility with motorized vehicles, is currently acquiring more importance, in order to improve the people quality of life studyng for example road mesh planning and communication resources allocation~\cite{herrera:10, Ma:2013}.

The first step in understanding urban mobility patterns is the proper acquisition of data. Data can be obtained in several ways, e.g., by observing vehicles passing through sensors or fixed/mobile radars, by acquiring traffic data from cameras, or by the active participation of users (\textit{crowdsourcing}). However, large and heterogeneous data acquisition is still a challenge. Indeed, only a few companies have access to them and usually, they embed some random components to protect the users' privacy ~\cite{ciociolaumap}. Therefore, it is important to collect and study the available open data and generate models that can help to understand the urban mobility and the social interactions of people in the urban environment.


Many alternative transport modes contribute to urban mobility. Among them, the car-sharing paradigm is quickly growing~\cite{boldrini:16,ciociolaumap,becker2017comparing}. In a car-sharing system, people can drive a vehicle, without worrying about buying it and paying for maintenance, fuel and parking fees. %These systems have already a significant volume of users and thus are an important type of urban mobility pattern. In fact,
By 2015, more than 1.5 million users and 22\,000 shared vehicles have been counted in the Americas, and growth in usage is still expected~\cite{shaheen2016mobility}. 
Overall, car-sharing services are classified into three categories: 
\begin{itemize}
		\item the one-way services, where the vehicles are available in specific stations and the user can move a car from a station to another
		\item the two-way services, where the user must return the vehicle to the same station she/he picked up the vehicle and;
		\item the free-floating service where vehicles are not tied to stations. In this case, the users are able to start and finish their trips everywhere within an operative area and in public parking spots~\cite{boldrini:16}
\end{itemize}. 

This chapter proposes a comparison between free floating and different station based car sharing paradigms. More in details we characterized those services in order to outstanding different users' habits. We take as case of study the city of Vancouver that hosts, several car-sharing providers. Our characterization relies on data we gathered for more than a year from Modo, Evo and Car2Go \footnote{service dismissed on February 29, 2020} car-sharing services ---a two-way, a one-way and a free-floating service, respectively---.
The chapter illustrates the users' demand and usage patterns of vehicles from these services and, at a glance, the contributions are twofold: first, a characterization of three important car-sharing paradigms and, second,  a demand model for their vehicles, providing statistical distributions which describe their busy and idle periods. 
This study is important to highlight particular situations where car-sharing services are attractive and, together with data from other transport modes, to uncover trends and mobility patterns. Moreover, we also believe the collected data and the developed models can be used to generate accurate synthetic workload. As a consequence, these can contribute to the development of better capacity planning models to car-sharing systems and also to a better plan of public transport systems.
%To best of our knowledge, we are the first to jointly consider all these three types of services, leveraging their common characteristics and highlighting their peculiarities.


The remainder of this chapter is structured as follows:  Section \ref{sec:4_5_related} describes related work; Section \ref{sec:4_2_carsharing} describes details of the three car-sharing paradigms; Section \ref{sec:4_3_methodology} discusses the data collection and analysis methodology for all services; Section \ref{sec:4_4_characterization} presents the results of the characterization for each model and the comparison of them, whereas Section \ref{sec:4_6_conclusion} concludes the paper.

%The urban mobility is in essence linked with human behavior and a wide number of urban activities. Because, it is the most common way to human patterns, like work relationships and social interactions, that have a huge value in the society life. Besides that, the data collecting methods brings the opportunity to understand better this patterns, that promotes academic studies and private applications for his understands and improvement. This studies are result of a growing demand of solutions that involves allocating the most diverse resources to its optimized operation. Consequently, understanding the network of relationships and human movements that are in urban mobility can assist to characterize many human patterns~\cite{barbosa:18,herrera:10}.}

% - Cultura do compartilhamento - estou decidindo se é relevante para a pesquisa
%At the same time, there is a growing interest in a faster and cheaper access even temporarily accessing a product instead of own them. The product sharing idea is already in our society, with libraries or taxi services. Furthermore, the internet ascension shows an opportunity to the sharing marketplace of cultural sources and services growth, instead of the classic idea of own a product and all the other costs that can bring with it. This can be seen in the current impact of sharing or on-demand services, as for entertainment (Netflix, Spotify) as for mobility (Uber, Zipcar)~\cite{botsman:10}. % posso citar livros?
%Therefore, has arisen a new consumer style, based on the access, it brings new research linking this concept with real-life cases~\cite{bardhi:12}. % Tenho que referenciar isso melhor

%There are many solutions for the best approach of urban mobility. Car-Sharing is a business model that has received increased attention. In this system, the user can schedule the utilization of a sharing vehicle for a similar use of a private car, without worrying with parking fees, maintenance or other taxes that a private car have. Previous research shows that these systems have already a large number of users and a great potential to be an important solution to urban mobility in comparison with other most common solutions~\cite{shaheen:16}.