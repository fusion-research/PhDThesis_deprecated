\section{Conclusions} 
\label{sec:3_6_conclusion}

In this article, we characterized three distinct car-sharing systems which operate in  Vancouver  (Canada)  and nearby regions. Our study, using data of more than one year of real trips, uncovers patterns of users’ habits.  We provided a characterization of the different car-sharing services, including spatial-temporal usage. Finally, we highlighted the main differences and the common characteristics of these services.

We showed that in Vancouver in 2017 the one-way and free-floating services were used similarly. They present shorter travels when compared to the two-way service. All three services present peaks of demand during the day. During working days, these peaks occur at around 8\,AM and 6\,PM, while in weekends, peaks are distributed in the afternoon. The two-way service we analyze presents a considerable number of booking cancellations and a higher vehicle idle time. This indicates a low utilization of the vehicles, likely due to their business model. Indeed, one-way and free-floating services allow users to pick-up a car and leave it anywhere in the city, dynamically satisfying the floating demand. 
We also highlight the strong relationship with the public transportation system, as well as with points of interests such as public universities and commercial centers.  Finally, we believe the characterization we provide may be used as a substrate for urban centers planning.

\section*{Acknowledgements}

The authors would like to thank CAPES, CNPq, FAPEMIG for their financial support in this research.