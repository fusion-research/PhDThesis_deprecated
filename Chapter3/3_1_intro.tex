\section{Introduction}
\label{sec:3_1_intro}

Mobility is one of the challenges to solve in our society and in cities, where eco-sustainability is becoming more and more important. 
Regulators and policy makers are positively looking into ``smart'' approaches to improve the current status of their urban network.  The ability to collect data, is the first step to take informed decisions. Unfortunately, getting information about mobility patterns and human driving habits is not easy because of both technical challenges and privacy issues.


Car sharing refers to a model of car rental where customers rent a car for a short period of time, usually for a few hours or less. One of its most interesting systems is the so called \textit{Free-Floating Car Sharing (FFCS)} system. The peculiarity of this system is that customers can pick and drop the car wherever in a geo-fence area.
The most famous company is car2go which is present in 25 cities and 8 different countries, both in Europe and North America.

To rent a car in a modern FFCS system, users check on their smartphone, or on the FFCS website, which cars are available in the neighborhood. Then, with a simple tap they can book a car, and start/end the rental. The FFCS app contacts a web-based back-end server to fetch data about available cars, perform a booking, and accounting operations. Typically for this purpose web API are used, some of which are publicly documented~\cite{car2goAPI}. The same website and app offer information about the status of the car rental systems, and the same web API can be used to collect for free this information. In the past, this approach has been successfully used to obtain data for specific mobility studies -- see Sec.~\ref{sec:3_2_related} for more details. This chapter focus the attention on the data collected in chap \ref{chap:2_dataset} by means of big data techniques to understand driving habits in a city. I take the city of Turin as a use case.

%I design \tool, a platform to collect, process, augment, and store data in a data lake, where analytics let the analyst extract higher level information.
%I build two crawlers to collect data from the \textit{car2go}  and \textit{Enjoy} platforms\footnote{\url{www.car2go.com}, \url{enjoy.eni.com}}, both present in Turin. Every minute, the crawler checks which cars are currently available. Every time a given car ``disappears'', it records the booking start time. The same booking ends when the crawler sees the car available back on the system. Some booking are actual ``rental'' in case the car moved from the prior parking position to another. Ingenuity must be used, e.g., to filter GPS fix issues (which may erroneously let a car ``move''), or to handle possible data collection issues (e.g., the website going down, or some cars undergoing in maintenance), or platform design (e.g., synchronous or asynchronous updates).

I let the crawlers described in chapter \ref{chap:2_dataset} for 52 days,  from December 10th 2016 to January 31st 2017, thus analysing the first stable chunk of data. In total \tool collected more than 104,000 \textit{bookings} and 86,000 \textit{rentals} for car2go, and 93,000 \textit{bookings} and 81,000 \textit{rentals} for Enjoy. 

With these datasets, I characterize the FFCS service utilization, in terms of bookings and rentals, with the aim to observe how people use these services, where they typically go, when, for how long the rental last, etc. Some observations are quite intuitive, e.g., people appear to be willing to use more the FFCS during weekdays and during peak-time. Counterintuitively, the rental duration and the driving distance show marginal changes over the day and weeks.

I complement the analysis by comparing the booking duration with the driving duration as suggested by Google Directions application, which the software can collect in real time for each rental. This highlights that 8.5\% of bookings last less than the Google driving time. 
This may be due to Google Directions overestimating the driving duration or, recalling that bookings  include the reservation time and the time to look for a parking spot, this may suggest that the time-based tariffs adopted by FFCS systems may encourage fast driving styles in the hope to reduce the rental cost.
I next compare the duration of the booking with the equivalent trip duration by public transport as returned again by Google Directions. I discover that rentals are 36\% shorter on average than public transport time, but  rentals start to be preferred when public transport time is higher than 10 minutes.

I presented the results of this analyses to the Turin Transportation Authority, who found them to be extremely useful to understand people driving habits. \tool may represents an important support tool for the investigation of car sharing users' habits. The scalable design of \tool allows the policy maker to collect data from many FFCS providers  and integrate it with other sources. This eases the analysis when taking in consideration trends and providers comparison. \tool allows the Transportation Authority to take informed decisions when planning public transport systems. This characteristic strengthens the potentiality of \tool for economical and sociological prediction and analysis. The data-driven approach, combined with other more traditional tools like surveys, represents an interesting observation point for understanding potential services improvements, both for car sharing and public transport systems. The source code of \tool  is available for research purposes.\footnote{\url{github.com/MobilityPolito/}}

The reminder of this chapter is structured as follows: Sec.~\ref{sec:3_2_related} discusses the related works. Sec.~\ref{sec:3_4_results} presents the in the following order: First, car2go and Enjoy car usage over time characterization; second, how customers drive the cars and how they move in the city; finally, users' driving habits and the correlation between booking time and the public transport time. Sec.~\ref{sec:3_5_conclu} concludes the chapter.


