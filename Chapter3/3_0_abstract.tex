Car sharing is nowadays a popular means of transport in smart cities. In particular, the free-floating paradigm lets the customers look for available cars, book one, and then start and stop the rental at their will, within a specific area. This is done thanks to a smartphone app, which contacts a web-based back-end to exchange information. 

In this chapter, I prove the flexibility of \tool, presented in chapter \ref{chap:2_dataset}, considering as  a case of study the city of Turin (Italy). I collect car sharing usage data for over 50 days to characterize both the temporal and spatial properties of rentals, and to characterize customers' habits in using the service, compared to public transportation  alternatives. Results provide insights about the driving style and needs, which are useful for smart city planners, and prove the feasibility of our approach.

This work refers mostly my paper " \textit{UMAP: Urban mobility analysis platform to harvest car sharing data}, presented in at 2017 IEEE SmartWord conference \cite{ciociolaumap}

