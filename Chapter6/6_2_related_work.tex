\section{Related work}
\label{sec:related}

The diffusion of the free floating approach to car sharing leaded to an increasing attention by many researchers, with  analyses of
these systems and their extension to electrical vehicles. 
The studies performed  in 2011 by Finkorn and M\"{u}ller \cite{Firnkorn2011,FM12} are the first attempts to analyze benefits of FFCS for the population. Their results on customers' behavior, like traveled distances, are similar to ours, and our first work~\cite{UMAP}.
Later works \cite{Car2GoGlobalAnalysis,CSandSocial,Schmoller2015} also collected data and analyzed the mobility patterns of customers and differences among cities.

The introduction of electrical vehicles for private and public transportation, brought the problem of placing the electric charging stations. Authors in \cite{ChargingStationForVehicularNetworks} show the benefits of placing charging stations with different power according to the customer parking duration. 
Few data driven studies address the charging station placement, either by respectively minimizing  cost of installation, power loss and maintenance  \cite{PlacementAndPowergrid,mipCSPpechino}, or by minimizing the customers' walked distances necessary to reach a charging pole\cite{placementAustin}. 

After a survey among FFCS customers in Ulm (Germany), authors of \cite{FM15}  investigate the positive influence and feasibility of an electric FFCS systems.
Lastly, authors of \cite{WB15} study the relocation of electric cars in FFCS, since few charging stations may be blocked by completely charged vehicles. 

Previously, in~\cite{taormina}, we performed several analysis about how to design an electrical FFCS in the city of Turin. In~\cite{maui} we introduced a first optimization of electric charging stations placement. In this work, we extend both works, by considering 4 cities as case study, and studying a new set of return policies to observe the impact of the willingness of customers to contribute to the system sustainability. We further extend our work by discussing the benefits of using charging hubs.

For our knowledge, in this work we are the first to validate a data driven approach for dimensioning an electric FFCS system, by analyzing and optimizing different metrics impacting customer experience in different cities word wide located.

%try to merge all the approaches coming from the previous works in order to design an electric free floating car sharing system. We will demonstrate that, using Turin as case of study, it is possible to obtain a charging station placement able to guarantee an car sharing services based on electric vehicles.


%, focus on the Then the authors of  started delving into study, enriching the analysis with a users' trip dataset. They characterized Car2go service, pointing out how the customers' habits are different among cities. 

%Instead, Kortum et al. \cite{CSandSocial} used as dataset the trip recorded from all the providers that they was able to collect. They provided a system trend description for different cities and they tried to infer on some deduction using data like population density, land use or median income in that cities.  
















