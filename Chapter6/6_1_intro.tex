\section{Introduction}
\label{sec:intro}

Nowadays mobility is a very important challenge for our society, with strong implications on pollution in large cities where more eco-sustainable solutions are positively seen as a mean to improve the current situation.
Along with the usage of public transport, the sharing mobility such as bike sharing, car pooling and car sharing, can help to address this problem. In this chapter,  I focus on the design of an electric car sharing system, where customers rent a car for moving within the city limits for short periods of time. I focus on the so called Free Floating Car Sharing (FFCS) system where customers are free to pick and return the car wherever they like, inside a geo-fenced area.
Electric car sharing systems need an infrastructure of recharging stations, whose design requires ingenuity~\cite{PlacementAndPowergrid,placementAustin,mipCSPpechino}. 
%Two are the main problems that need to be faced: i) the charging station placement problem, i.e., how many and where to install charging stations; and ii) the return policy customers have to follow at the end of the rental, i.e., in which cases asking the customer to return the car to a charging station.

Data is fundamental to answer these design questions. In this work, I base the study on the availability of millions of actual rentals I collected from currently in use FFCS systems as reported in chapter \ref{chap:2_dataset}. 
%Here, I generalize the approach considering Turin and Vancouver (charchetrized in chapters \ref{chap:3_charact} and \ref{chap:4_cs_comparison}) adding to other cities in different countries, while extending and deepening our findings. In this paper, we leverage real FFCS data related to the city of Vancouver (Canada), Berlin (Germany), Milan, and Turin (Italy).
Im this chapter I consider Turin and Vancouver (charchetrized in chapters \ref{chap:3_charact} and \ref{chap:4_cs_comparison}) plus Berlin(Germany) and Milan(Italy) which in the dataset are the cities having the biggest number of rentals.
The data naturally factors the non-stationarity of FFCS systems including millions of actual rentals. Armed with this, I study and compare the performance of a hypothetical equivalent car sharing system based on electric vehicles.  While in the past some works have proposed solutions for the design of electric FFCS~\cite{FM15,WB15} and for a smart placement of charging stations, we are among the first to take a data driven approach for the design of electric car FFCS systems~\cite{ChargingStationForVehicularNetworks,mipCSPpechino,PlacementAndPowergrid,placementAustin}.

First, I characterize how customers actually use FFCS transport means in different cities and countries. Results show a similar usage with an high utilization during commuting time and very different spatial distributions. Rental duration and driving distance are quite short (less than 20-30 minutes, for less than 5\,km in median). More interestingly, it is possible to observe peripheral zones where cars are left parked for long time, and busy areas where instead the parking duration is much short and dynamic.

Armed with these facts, I compare charging station placement policies introduced in section \ref{sec:5_3_mh_placement} that exploit the knowledge of typical parking zones and duration. I first assume a pure Free Floating system, where customers return the car in a charging station only if present at their actual destination.
Results show that placing the charging stations in those areas where cars stay parked for long time performs badly. Instead, placing charging station in those areas where cars are frequently parked and rented, e.g., near train stations and working areas, guarantees much better performance. This is consistent in all cities.

Next, I study different return policies introduced in section \ref{sec:5_4_return_policy}, where customers are asked to return the car to a charging station in case the battery level decreases below a minimum threshold. This collaborative policy reduces the cost of the charging infrastructure by a factor of 2 or more with respect to a pure opportunistic free floating solutions. Equipping just 8\% of charging zones with 4 poles of 2\,kW would guarantee a electric car FFCS equivalent to the one currently in use.
This with minimal impact on the customer satisfaction, measured by the number of times customers are forced to drive to a charging station and the distance they have to walk back to the desired destination.

At last, I compare system design alternatives to check whether is better to place a lot of charging poles in very few areas, or rather to spread a lot charging stations with few poles in many areas. Results demonstrate that both extreme solutions perform badly, with best performance when installing charging stations with  5 to 20 poles in popular areas. This has benefits also on the power grid used to supply power to the recharging areas.

%We believe results presented in this paper, guided by actual usage pattern for FFCS customers, are very important for regulators and policy makers, as well as for researchers working in this area. In an effort to allow reproducibility and extend our results, we make both dataset and simulator publicly available.\footnote{\url{https://github.com/michelelt/sim3.0}}

After quickly discussing related work in Section~\ref{sec:related}, we present and characterize data in Section~\ref{sec:data}, and the simulation model and tool in Section~\ref{sec:Modelling}. Section~\ref{sec:freefloating} discusses the impact of charging stations placement policies, while Section~\ref{sec:results} compares return policies. Section~\ref{sec:resPoles} presents the impact of concentrating or spreading charging stations in the city. Finally Section~\ref{sec:conclusion} concludes the paper.
