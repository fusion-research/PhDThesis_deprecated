Car sharing is nowadays a popular means of transport in smart cities. In particular, the free-floating paradigm lets the customers look for available cars, book one, and then start and stop the rental at their will, within a specific area. This is done thanks to a smartphone app, which contacts a web-based backend to exchange information. In this paper we present \tool, a platform to harvest the data freely made available on the web by these backends and to extract driving habits in cities.

We design \tool with two specific purposes. Firsty \tool fetches data from car sharing platforms in real time. Secondly, it processes the data to extract advanced information about driving patterns and user's habits. To extract information, \tool augments the data available from the car sharing platforms with mapping and direction information fetched from other web platforms. This information is stored in a data lake where historical series are built, and later analyzed using analytics modules easy to design and customize. 

We prove the flexibility of \tool by presenting a case of study for the city of Turin. We collect car sharing usage data for over 50 days to  characterize both the temporal and spatial properties of rentals, and to characterize customers' habits in using the service, which we contrast with public transportation  alternatives. Results provide insights about the driving style and needs, which are useful for smart city planners, and prove the feasibility of our approach.
