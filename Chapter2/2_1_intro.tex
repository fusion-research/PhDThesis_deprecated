\section{Introduction}
\label{sec:intro}

Mobility is one of the challenges to solve in our society and in cities, where eco-sustainability is becoming more and more important. 
Regulators and policy makers are positively looking into ``smart'' approaches to improve the current status of their urban network.  The ability to collect data, is the first step to take informed decisions. Unfortunately, getting information about mobility patterns and human driving habits is not easy because of both technical challenges and privacy issues.

To this extent, in this paper we investigate the possibility of harvesting data openly exposed on the Web to obtain information about mobility habits in cities, and make it available to the players by using a smart-platform. We focus on car sharing platforms and mapping and direction services.
%
%For many years the usage of a private car for transportation was the most important solution. Nowadays this trend seems to decrease, especially among young adults for who owning a private car look to be less important. This change to alternative mobility services favorites the growth of car sharing systems.
%
%The first known solution of such a system is the Selbstfahrergenossenschaft car  sharing program born in 1948 in Zurich without a former development of it. Only in the early 1970s saw a first system based on car sharing. However, only recently, with the growth of the sharing economy, the car sharing system accelerated its growth, making nowadays the car sharing widely used in different parts of the world. As a matter of fact, the importance of car sharing systems is demonstrated by the expansion in term of mermership number. According to~\cite{a,b} this growth exceeds 10\% per year either in North America and Germany.

%Among the key advantages of such a system, has been proved that it allows traffic and pollution reduction~\cite{}.

Car sharing refers to a model of car rental where customers rent a car for a short period of time, usually for a few hours or less. One of its most interesting systems is the so called \textit{Free-Floating Car Sharing (FFCS)} system. The peculiarity of this system is that customers can pick and drop the car wherever in a geo-fence area.
The most famous company is car2go which is present in 25 cities and 8 different countries, both in Europe and North America.

%, where customers rent a car via the web or a smartphone app
% In the station-based approach, users have to pick-up the car and release it in specific positions e.g., parkings spot reserved for this car. A first example of this system is the famous American \textit{Zipcar} service, which allows users to pick up the car in specific parking spots and then return the car in the same starting point. An alternative is the one-way solution, like  \textit{Autolib'} in Paris, an electrical car sharing system where users can rent a car in one station and return to a different one. The peer-to-peer solution is based on the idea that most private car are most of the time unused(90\% of the time according to some estimation~\cite{c,yltechrep}). In this model, cars are made available by private users. The company which manage this kind of service only takes care of providing the insurance and finding customers to rent the car. This system is quite new to the market and it is exploited by a few companies (the most important one is RelayRides, in the US).
%
To rent a car in a modern FFCS system, users check on their smartphone, or on the FFCS website, which cars are available in the neighborhood. Then, with a simple tap they can book a car, and start/end the rental. The FFCS app contacts a web-based backend server to fetch data about available cars, perform a booking, and accounting operations. Typically for this purpose web API are used, some of which are publicly documented~\cite{car2goAPI}. The same website and app offer information about the status of the car rental systems, and the same web API can be used to collect for free this information. In the past, this approach has been successfully used to obtain data for specific mobility studies -- see Sec.~\ref{sec:related} for more details.
In this work, we extend this idea and focus our attention on the acquisition and harvesting of this data by means of big data techniques to understand driving habits in a city. We take the city of Turin as a use case.

We design \tool, a platform to collect, process, augment, and store data in a data lake, where analytics let the analyst extract higher level information.
We build two crawlers to collect data from the \textit{car2go}  and \textit{Enjoy} platforms\footnote{\url{www.car2go.com}, \url{enjoy.eni.com}}, both present in Turin. Every minute, the crawler checks which cars are currently available. Every time a given car ``disappears'', it records the booking start time. The same booking ends when the crawler sees the car available back on the system. Some booking are actual ``rental'' in case the car moved from the prior parking position to another. Ingenuity must be used, e.g., to filter GPS fix issues (which may erroneously let a car ``move''), or to handle possible data collection issues (e.g., the website going down, or some cars undergoing in maintenance), or platform design (e.g., synchronous or asynchronous updates).

We let the crawler run to collect data for 52 days, from December 10th 2016 to January 31st 2017. We observed more than 104,000 \textit{bookings} and 86,000 \textit{rentals} for car2go, and 93,000 \textit{bookings} and 81,000 \textit{rentals} for Enjoy. 
%Then, we characterize how user booked or rented the car during such a period, discovering that during the business day FFCS systems are used much more with respect to the weekend. To have a better understating on what is the trend during these two parts of the week, we then characterize the rentals dividing the week in two and studying the trend per hour. We discovered that while during weekdays the system is quite unused during the night, during the weekend we observe twice as much rents with respect to during the week days, but with a reduction during the day especially in the morning. Then, we characterize for how long time users rent cars and how far they go. Interestingly, we first discovered that despite the different pattern between week days and weekend days, the time and the distance users use to use a car does not change. Secondly, we observed that users using car2go use the car also for longer trip due to the possibility to drop the car in the Airport parking lot. By analyzing the duration time in case of bookings without a rental instead, we observed that the higher probability is less then 5 minutes and then at 20 minutes for car2go and around 15 minutes for Enjoy. While the first can be explained by a system fault, the second cases depends on the maximum \textit{reservation time} i.e., the maximum time an user can reserve a car without paying any fee. 
With these datasets, we characterize the FFCS service utilization, in terms of bookings and rentals, with the aim to observe how people use these services, where they typically go, when, for how long the rental last, etc. Some observations are quite intuitive, e.g., people appear to be willing to use more the FFCS during weekdays and during peak-time. Counterintuitively, the rental duration and the driving distance show marginal changes over the day and weeks.

%When comparing the two platforms, we discover that car2go users \DG{sometimes} drive the car for longer trips than the Enjoy users. This is partially due to the possibility to reach the Airport with car2go but not for Enjoy. We also observe that the usage increases during weekend evenings. Both effects suggest a lack of public transport which may be considered by the municipality for improvements.

% By analyzing the duration time of all the bookings that do not describe a rental, we observe that the highest percentage of them have a duration equal or lower than 5 minutes,  other common values are 20 minutes for car2go and around 15 minutes for Enjoy. While the shortest interval can be explained by a system fault or a booking cancellation, the other durations time can be linked with the maximum \textit{reservation time} i.e., the maximum time an user can reserve a car without paying any fee. 

%\DG{MESSAGGIO HEATMAP TBD}

We complement the analysis by comparing the booking duration with the driving duration as suggested by Google Directions application, which we collect in real time for each rental. This allows us to find that 8.5\% of bookings last less than the Google driving time. 
This may be due to Google Directions overestimating the driving duration or, recalling that bookings  include the reservation time and the time to look for a parking spot, this may suggest that the time-based tariffs adopted by FFCS systems may encourage fast driving styles in the hope to reduce the rental cost.
We next compare the duration of the booking with the equivalent trip duration by public transport as returned again by Google Directions. We discover that rentals are 36\% shorter on average than public transport time, but  rentals start to be preferred when public transport time is higher than 10 minutes.

We presented our results to the Turin Transportation Authority, who found them to be extremely useful to understand people driving habits. We believe that \tool represents an important support tool for the investigation of car sharing users' habits. The scalable design of \tool allows the policy maker to collect data from many FFCS providers  and integrate it with other sources. This eases the analysis when taking in consideration trends and providers comparison. \tool allows the Transportation Authority to take informed decisions when planning public transport systems. This characteristic strengthens the potentiality of \tool for economical and sociological prediction and analysis. Our data-driven approach, combined with other more traditional tools like surveys, represents an interesting observation point for understanding potential services improvements, both for car sharing and public transport systems. We make available the source code of \tool for research purposes.\footnote{\url{github.com/MobilityPolito/}}

The reminder of this paper is structured as follows: Sec.~\ref{sec:related} discusses the related work. 
Sec.~\ref{sec:methodology} describes in details \tool data acquisition and analysis capabilities. Sec.~\ref{sec:results} presents our results: First, we characterize car2go and Enjoy car usage over time; second, how customers drive the cars and how they move in the city; finally, we show what are the users' driving habits and the correlation between booking time and the public transport time. Sec.~\ref{sec:conclu} concludes the paper.


