\section{Conclusion}\label{key}
\label{sec:5_6_conclusion}
In this chapter I described a FFCS electric mobility simulator I developed. Starting from the data collected with the software described in chapter \ref{chap:2_dataset} I created trace of rental events, describing the system allocated users' demand. 

More in details, the simulator allocates a set of cars, characterized by battery capacity and power consumption per kilometres. Then consumes the rental trace, marking the car unavailable after a \textit{rental-start} event and updating the final battery state of charge when a \textit{rental-end} event is processed. Moreover, the simulator is in charge to place the charging station according three heuristics: random, preferring zones having a grater parking time and zones having the higher number of parkings. Finally it takes in account the different policies with which the users have to return the car.

When the trace is consumed, this simulator computes several key performance indicators measuring the proper system infrastructure allocation and users' discomfort to deal with an electric vehcile